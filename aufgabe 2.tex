\documentclass[10pt, a4paper]{article}

\title{CS 102 Latex \"Ubung}
\author{Luisa Kral}
\date{05.11.2014}

\begin{document}
\maketitle
\section{Pr\"aambel}
Hier beginnt die Einleitung..

\section{Tabelle}
\begin{table}[h]
\centering
\begin{tabular}{|l|l|r|}
\hline
\textbf{Aufgaben} & \textbf{Punkte erhalten} & \textbf{Punkte m\"glich}  \\
\hline
Aufgabe 1 & 8 & 10  \\
Aufgabe 2 & 4 & 10  \\
Aufgabe 3 & 2 & 10  \\
\hline
\end{tabular}
\caption{Formen}
\label{tab:formen}
\end{table}

\section{3.Abschnitt}
\begin{equation}
\label{abc}
\subsection{Satz des Pythagoras}
Der Satz des Pythagoras errechnet sich wie folgt:$a^2+b^2=c^2$. Daraus kann man die L\"ange der Hypotenuse wie folgt berechnen: $c= \sqrt{a^2+b^2}$.
\subsection{Summen}
Man kann auch die Formel f\"ur eine Summe angeben:
$s= \sum\limits_{i=1}^{n}i=(n*(n+1))/2$
\end{equation}

\end{document}